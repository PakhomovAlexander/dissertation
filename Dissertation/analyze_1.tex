\chapter{Анализ существующих подходов} \label{chapt_1}

\todo{
Первая глава формируется на основе изучения имеющейся отече- ственной и зарубежной научной и специальной литературы по исследуемой теме (с обязательными ссылками на источники!), а также нормативных ма- териалов. В ней содержится описание объекта и предмета исследования по- средством различных теоретических концепций, принятых понятий и их классификации, а также степени проработанности проблемы в России и за ее пределами.
Автор должен продемонстрировать глубину погружения в проблему, владение знаниями о текущем состоянии ее решения путем анализа макси- мально возможного количества источников. В редкой ситуации полной новизны, тем не менее, необходимо проанализировать состояние выбранной предметной области с последующими выводами об актуальности заявлен- ных исследований.
В первой главе могут рассматриваться существующие подходы к ре- шению задач исследования, проводиться их сравнительный анализ с ис- пользованием системы критериев. Результаты анализа могут быть пред- ставлены в виде таблиц, графиков, диаграмм, схем для того, чтобы сделать выводы о сильных и слабых сторонах имеющихся решений и обосновать собственные предложения и подходы.
Кроме того, может быть предложен собственный понятийный аппарат (при необходимости).
Первая глава, по сути, служит теоретическим обоснованием исследо- ваний, проведенных автором.
Последующие главы магистерской диссертации строятся по схеме: математическое, алгоритмическое, программное обеспечение.
}

