\chapter{Анализ существующих подходов к генерации тестового кода} \label{chapt_1}

\todo{
Первая глава формируется на основе изучения имеющейся отечественной и зарубежной научной и специальной литературы по исследуемой теме (с обязательными ссылками на источники!), а также нормативных материалов. В ней содержится описание объекта и предмета исследования посредством различных теоретических концепций, принятых понятий и их классификации, а также степени проработанности проблемы в России и за ее пределами.
Автор должен продемонстрировать глубину погружения в проблему, владение знаниями о текущем состоянии ее решения путем анализа максимально возможного количества источников. В редкой ситуации полной новизны, тем не менее, необходимо проанализировать состояние выбранной предметной области с последующими выводами об актуальности заявленных исследований.
В первой главе могут рассматриваться существующие подходы к решению задач исследования, проводиться их сравнительный анализ с ис- пользованием системы критериев. Результаты анализа могут быть пред- ставлены в виде таблиц, графиков, диаграмм, схем для того, чтобы сделать выводы о сильных и слабых сторонах имеющихся решений и обосновать собственные предложения и подходы.
Кроме того, может быть предложен собственный понятийный аппарат (при необходимости).
Первая глава, по сути, служит теоретическим обоснованием исследований, проведенных автором.
Последующие главы магистерской диссертации строятся по схеме: математическое, алгоритмическое, программное обеспечение.
}


\section{Введение в тестирование программного обеспечения} \label{section_1}

\fixme{TBD}

\section{Лексическая генерация случайных входных данных} \label{section_2}

\subsection{Fuzzing: Breaking Things with Random Inputs}
\fixme{TBD}


\subsection{Code Coverage}
\fixme{TBD}


\subsection{Mutation-Based Fuzzing}
\fixme{TBD}


\subsection{Greybox Fuzzing}
\fixme{TBD}


\subsection{Search-Based Fuzzing}
\fixme{TBD}


\subsection{Mutation Analysis}
\fixme{TBD}


\section{Синтаксическая генерация случайных входных данных} \label{section_3}

\subsection{Fuzzing with Grammars}
\fixme{TBD}


\subsection{Efficient Grammar Fuzzing}
\fixme{TBD}


\subsection{Grammar Coverage}
\fixme{TBD}


\subsection{Parsing Inputs}
\fixme{TBD}


\subsection{Probabilistic Grammar Fuzzing}
\fixme{TBD}


\subsection{Fuzzing with Generators}
\fixme{TBD}

\subsection{Greybox Fuzzing with Grammars}
\fixme{TBD}

\subsection{Reducing Failure-Inducing Inputs}
\fixme{TBD}


\section{Семантическая генерация случайных входных данных} \label{section_4}

\subsection{Mining Input Grammarss}
\fixme{TBD}


\subsection{Tracking Information Flow}
\fixme{TBD}


\subsection{Concolic Fuzzing}
\fixme{TBD}


\subsection{Symbolic Fuzzing}
\fixme{TBD}


\subsection{Mining Function Specifications}
\fixme{TBD}



\section{Доменная генерация случайных входных данных} \label{section_4}

\subsection{Testing Configurations}
\fixme{TBD}


\subsection{Fuzzing APIs}
\fixme{TBD}


\subsection{Carving Unit Tests}
\fixme{TBD}


\subsection{Testing Web Applications}
\fixme{TBD}


\subsection{Testing Graphical User Interfaces}
\fixme{TBD}

