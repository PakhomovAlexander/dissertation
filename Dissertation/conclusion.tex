\chapter*{Заключение}                       % Заголовок
\addcontentsline{toc}{chapter}{Заключение}  % Добавляем его в оглавление

\hspace*{2.5em}В результате работы были проанализированы существующие инструменты автогенерации модульных тестов. Разработан инструмент автогенерации модульных тестов для Java программ. Инструмент удовлетворяет следующим требованиям:

\begin{itemize}
	\item поставка в виде отдельной библиотеки;
	\item генерация соответствующего модульного теста или нескольких модульных тестов для указанного Java класса;
	\item набор тестовых сценариев обеспечивает максимально возможное покрытие по критерию покрытия веток исполнения;
	\item сгенерированный код понятен человеку.
\end{itemize}


\textbf{Публикация результатов.} Материалы по~исследованию техник модульного тестирования были опубликованы в~статье:  Брусенцев~И.М., Пахомов~А.С.  Анализ техники ведения разработки через тестирование на~языке Java // Актуальные проблемы прикладной математики, информатики и~механики : сборник трудов международной научной конференции, Воронеж, 7-9~декабря.~-- Воронеж,~2020.~-- C.~601-603; URL: http://www.amm.vsu.ru/conf/index.php?page=Doklads (дата обращения: 12.06.2021).
