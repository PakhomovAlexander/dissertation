\chapter*{Заключение}                       % Заголовок
\addcontentsline{toc}{chapter}{Заключение}  % Добавляем его в оглавление

\fixme{
В заключении логически последовательно излагаются теоретические и практические выводы, результаты и предложения, которые получены в результате исследования. Они должны быть краткими, четкими, дающими полное представление о содержании, значимости, обоснованности и эффек- тивности исследований и разработок.
Кроме того, в заключении можно представить практическую значи- мость и результаты реализации работы, подразумевающие разработку ма- тематического, алгоритмического, программного обеспечения для решения определенной задачи или класса задач, наличие внедрения в учебный, ис- следовательский, производственный процесс, регистрацию программных средств, наличие патента, рекомендации к использованию.
В заключении приводится список публикаций автора и апробация ра- боты на конференциях различного уровня.
}


\textbf{Публикация результатов.} Материалы по~исследованию техник модульного тестирования были опубликованны в~статье:  Брусенцев~И. М.,~Пахомов~А.С.  Анализ техники ведения разработки через тестирование на~языке Java // Актуальные проблемы прикладной математики, информатики и~механики : сборник трудов международной научной конференции, Воронеж, 7-9~декабря.~--Воронеж,~2020.~-- C.~601-603; URL: http://www.amm.vsu.ru/conf/index.php?page=Doklads (дата обращения: 12.06.2021).
