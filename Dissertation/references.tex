\clearpage                                  % В том числе гарантирует, что список литературы в оглавлении будет с правильным номером страницы
%\hypersetup{ urlcolor=black }               % Ссылки делаем чёрными
%\providecommand*{\BibDash}{}                % В стилях ugost2008 отключаем использование тире как разделителя 
\urlstyle{rm}                               % ссылки URL обычным шрифтом
\ifdefmacro{\microtypesetup}{\microtypesetup{protrusion=false}}{} % не рекомендуется применять пакет микротипографики к автоматически генерируемому списку литературы
%\insertbibliofull                           % Подключаем Bib-базы

\chapter*{Список литературы}          

\addcontentsline{toc}{chapter}{Список литературы}

\begin{enumerate}

\item https://ru.wikipedia.org/wiki/Тестирование\_программного\_обеспечения
	
\item https://ru.wikipedia.org/wiki/Ручное\_тестирование

\item https://ru.wikipedia.org/wiki/Автоматизированное\_тестирование

\item https://junit.org/junit5/

\item Zhu, H., Hall, P. A.,\& May, J. H. (1997). Software unit test coverage and adequacy. ACM computing surveys (csur), 29(4), 366-427.

\item https://ru.wikipedia.org/wiki/Логика\_Хоара#Тройки\_Хоара

\item https://en.wikipedia.org/wiki/QuickCheck

\item https://jqwik.net

\item https://sttp.site/chapters/intelligent-testing/mutation-testing.html

\item http://pitest.org

\item https://ru.wikipedia.org/wiki/Фаззинг

\item http://www0.cs.ucl.ac.uk/staff/mharman/ACM-surveys-sbse.pdf

\item https://mcminn.io/publications/c18.pdf

\item https://randoop.github.io/randoop/manual/dev.html#starting

\item https://www.evosuite.org

\item https://www.diffblue.com

\item https://engineering.fb.com/2018/05/02/developer-tools/sapienz-intelligent-automated-software-testing-at-scale/


\end{enumerate}


\ifdefmacro{\microtypesetup}{\microtypesetup{protrusion=true}}{}
\urlstyle{tt}                               % возвращаем установки шрифта ссылок URL
%\hypersetup{ urlcolor={urlcolor} }          % Восстанавливаем цвет ссылок