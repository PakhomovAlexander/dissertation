\clearpage                                  % В том числе гарантирует, что список литературы в оглавлении будет с правильным номером страницы
%\hypersetup{ urlcolor=black }               % Ссылки делаем чёрными
%\providecommand*{\BibDash}{}                % В стилях ugost2008 отключаем использование тире как разделителя 
\urlstyle{rm}                               % ссылки URL обычным шрифтом
\ifdefmacro{\microtypesetup}{\microtypesetup{protrusion=false}}{} % не рекомендуется применять пакет микротипографики к автоматически генерируемому списку литературы
%\insertbibliofull                           % Подключаем Bib-базы

\chapter*{Список литературы}          

\addcontentsline{toc}{chapter}{Список литературы}

\begin{enumerate}

\item https://ru.wikipedia.org/wiki/Тестирование_программного_обеспечения
	
\item https://ru.wikipedia.org/wiki/Ручное_тестирование

\item https://ru.wikipedia.org/wiki/Автоматизированное_тестирование

\item https://junit.org/junit5/

\item Zhu, H., Hall, P. A.,\& May, J. H. (1997). Software unit test coverage and adequacy. ACM computing surveys (csur), 29(4), 366-427.

\item https://ru.wikipedia.org/wiki/Логика_Хоара#Тройки_Хоара

\end{enumerate}


\ifdefmacro{\microtypesetup}{\microtypesetup{protrusion=true}}{}
\urlstyle{tt}                               % возвращаем установки шрифта ссылок URL
%\hypersetup{ urlcolor={urlcolor} }          % Восстанавливаем цвет ссылок