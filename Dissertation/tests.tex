\section{План тестирования} \label{sub26}

Для проверки корректности работы инструмента необходимо провести ряд тестов.

\textbf{Тест 1. <<Генерация модульного теста для класса без конструктора с одним публичным методом без параметров>>}

Цель: проверить корректность работы инструмента  для класса без~конструктора с одним публичным методом без параметров.

Порядок проведения: создать java класс без конструктора с одним публичным методом без параметров, запустить инструмент.

Результат: сгенерирован модульный тест, который достигает полного покрытия.


\textbf{Тест 2. <<Генерация модульного теста для класса без~конструктора с одним публичным методом c одним параметром типа int>>}

Цель: проверить корректность работы инструмента  для класса без конструктора с одним публичным методом с одним параметром типа int.

Порядок проведения: создать java класс без конструктора с одним публичным методом с одним параметров типа int, запустить инструмент.

Результат: сгенерирован модульный тест, который достигает полного покрытия.


\textbf{Тест 3. <<Генерация модульного теста для класса без~конструктора с одним публичным методом c двумя параметрами типа int>>}

Цель: проверить корректность работы инструмента  для класса без конструктора с одним публичным методом с двумя параметрами типа int.

Порядок проведения: создать java класс без конструктора с одним публичным методом с двумя параметрами типа int, запустить инструмент.

Результат: сгенерирован модульный тест, который достигает полного покрытия.

\textbf{Тест 4. <<Генерация модульного теста для класса с одним конструктором и с одним публичным методом без параметров>>}

Цель: проверить корректность работы инструмента для с одним конструктором и с одним публичным методом без параметров.

Порядок проведения: создать java класс с одним конструктором и~с~одним публичным методом без параметров, запустить инструмент.

Результат: сгенерирован модульный тест, который достигает полного покрытия.


\textbf{Тест 5. <<Генерация модульного теста для класса с одним конструктором и с одним публичным методом с одним параметром типа String>>}

Цель: проверить корректность работы инструмента для с одним конструктором и с одним публичным методом с одним параметром типа String.

Порядок проведения: создать java класс с одним конструктором и~с~одним публичным методом с одним параметром типа String, запустить инструмент.

Результат: сгенерирован модульный тест, который достигает полного покрытия.


\textbf{Тест 6. <<Корректное завершение работы при ошибке пользователя>>}

Цель: проверить корректность работы инструмента в случае указания пользователем несуществующего класса.

Порядок проведения: запустить инструмент с указанием несуществую-щего класса.

Результат: вывод ошибки в стандартный поток вывода.