\chapter*{Введение}                         % Заголовок
\addcontentsline{toc}{chapter}{Введение}    % Добавляем его в оглавление


\hspace*{2.5em}Разработка программного обеспечения включает в себя процесс тестирования. Самым эффективным способом тестирования на сегодняшний день считается автоматизированное тестирование. Чаще всего автомати-зируется сам запуск тестовых сценариев. Процесс анализа и написания тестов производится человеком.

\textbf{Актуальность работы} обусловлена необходимостью производства качественного программного обеспечения с минимальным участием человека в процессе тестирования.

\textbf{Предметом} изучения является тестирование программного обеспечения с помощью искусственного интеллекта. 

\textbf{Целью} данной работы является исследование методов тестирования программного обеспечения с~помощью искусственного интеллекта и~разработка инструмента автогенерации модульных тестов из~Java кода.​

\textbf{Практическая значимость} работы обусловлена возможностью применения разработанного инструмента в коммерческих программных комплексах, а так же в программном обеспечении с открытым исходным кодом. Автоматизированный процесс генерации модульных тестов повышает качество разрабатываемого программного обеспечения без~участия человека.

%\newcommand{\actuality}{}
%\newcommand{\progress}{}
%\newcommand{\aim}{{\textbf\aimTXT}}
%\newcommand{\tasks}{\textbf{\tasksTXT}}
%\newcommand{\novelty}{\textbf{\noveltyTXT}}
%\newcommand{\influence}{\textbf{\influenceTXT}}
%\newcommand{\methods}{\textbf{\methodsTXT}}
%\newcommand{\defpositions}{\textbf{\defpositionsTXT}}
%\newcommand{\reliability}{\textbf{\reliabilityTXT}}
%\newcommand{\probation}{\textbf{\probationTXT}}
%\newcommand{\contribution}{\textbf{\contributionTXT}}
%\newcommand{\publications}{\textbf{\publicationsTXT}}

%\textbf{Объем и структура работы.}
%% на случай ошибок оставляю исходный кусок на месте, закомментированным
%Полный объём диссертации составляет  \ref*{TotPages}~страницу
%с~\totalfigures{}~рисунками и~\totaltables{}~таблицами. Список литературы
%содержит \total{citenum}~наименований.
%
%Полный объём дипломной работы составляет
%\formbytotal{TotPages}{страниц}{у}{ы}{}, включая
%\formbytotal{totalcount@figure}{рисун}{ок}{ка}{ков} и
%\formbytotal{totalcount@table}{таблиц}{у}{ы}{}.   Список литературы содержит
%\formbytotal{citenum}{наименован}{ие}{ия}{ий}.
