\section{Введение в тестирование программного обеспечения} \label{section_1}

Тестирование программного обеспечения~--- процесс исследования, испытания программного продукта, имеющий своей целью проверку соответствия между реальным поведением программы и её ожидаемым поведением на конечном наборе тестов, выбранных определённым образом~[1]. Существеут множество техник и подходов к тестированию программноо обеспечения.  

\subsection{Ручное тестирование} \label{subsection_11}

Ручное тестирование (англ.~manual~testing)~--- часть процесса тестирования на этапе контроля качества в процессе разработки ПО. Оно производится тестировщиком без использования программных средств, для проверки программы или сайта путём моделирования действий пользователя~[2]. 

Приемущества такого способа тестирования:

\begin{itemize}
	\item Простота. От тестировщика не требуется знания специальных инструментов атоматизации.
	\item Тестируется именно то, что видет пользователь.
\end{itemize} 

Основные проблемы ручного тестирования:
\begin{itemize}
	\item Наличие человеческого труда. Тестировщих может допустить ошибку в процессе ручных действий.
	\item Выполнение ручных дейсвий может занимать много времени.
	\item Такой вид тестирования не способен покрыть все сценарии использования ПО. 
	\item Не исключается повторное внесение ошибки. Если пользователь системы нашел ошибку, тестировщик воспроизведет её только один раз. В последующих циклах разработки ПО ошибка может быть внесена повторно.
\end{itemize} 


\subsection{Автоматизированное тестирование} \label{subsection_12}
 
 Автоматизированное тестирование программного обеспечения~--- часть процесса тестирования на этапе контроля качества в процессе разработки программного обеспечения. Оно использует программные средства для выполнения тестов и проверки результатов выполнения~[3].
 
 Подходы к автоматизации тестирования:
 
 \begin{itemize}
 	\item Тестирование пользовательского интерфейса. С помощью специальных тестовых библиотек производится имитация действий пользователя.
 	\item Тестирование на уровне кода (модульное тестирование).
 \end{itemize} 

 Приемущества атоматизированного тестирования:

\begin{itemize}
	\item сокращение времени тестирования; 
	\item уменьшение вероятности допустить ошибку по сравнению с ручным тестированием;
	\item исключение появления ошибки в последующей разработки программного обеспечиния.
\end{itemize} 

 Недостатки атоматизированного тестирования:

\begin{itemize}
	\item Трудоемкость. Поддержка и обновление тестов являются трудоемким процессом.
	\item Необходимость знания инструментария.
	\item Автоматическое тестирование не может полностью заменить ручное. На практике используется комбинация ручного и автоматизированного тестирования.
\end{itemize} 

Существует множество иструментов для написания и запуска тестов на языке Java: JUnit, Spock Framework, TestNG, UniTESK, JBehave, Serenity, Selenide, Gauge, Geb.


\subsubsection{JUnit}

JUnit~--- самый распространненый инструмент для написания и запуска тестов на языке Java. Последняя версия 5.7.1~[4].

Сценарий использования JUnit 5: 

\begin{enumerate}
	\item Определить тестируемый класс или модуль. Листинг~1.1.
	\item Создать новый класс, для написания тестов. По соглашению, имя класса должно совпадать с именем тестируемого класса и заканчиваться постфиксом \textit{Test}. Листинг~1.2.
	\item Для каждого тестового сценария необходимо написать метод и пометить его аннотацией \textit{@org.junit.jupiter.api.Test}.
	\item В каждом сценарии нужно написать соответствующий код, который заканчивается выражением из пакета \textit{org.junit.jupiter.api.Assertions.*}. 
	\item Запустить тест в среде разработки (IDE) или с помощью системы сборки (Gradle, Maven).
\end{enumerate}

\begin{ListingEnv}[!h]% настройки floating аналогичны окружению figure
	\captiondelim{ } % разделитель идентификатора с номером от наименования
	\caption{Тестируемый класс \textit{RomanNumeral}}
	% окружение учитывает пробелы и табуляции и применяет их в сответсвии с настройками
	\begin{lstlisting}[language={Java}]
	public class RomanNumeral {
		private static Map<Character, Integer> map;
		
		static {
			map = new HashMap<>();
			map.put('I', 1);
			map.put('V', 5);
			map.put('X', 10);
			map.put('L', 50);
			map.put('C', 100);
			map.put('D', 500);
			map.put('M', 1000);
		}
		
		public int convert(String s) {
			int convertedNumber = 0;
			
			for (int i = 0; i < s.length(); i++) {
				int currentNumber = map.get(s.charAt(i));
				int next = i + 1 < s.length() ? map.get(s.charAt(i + 1)) : 0;
				
				if (currentNumber >= next) {
					convertedNumber += currentNumber;
				} else {
					convertedNumber -= currentNumber;
				}
			}
			
			return convertedNumber;
		}
	}
	\end{lstlisting}
\end{ListingEnv}%

\begin{ListingEnv}[!h]% настройки floating аналогичны окружению figure
	\captiondelim{ } % разделитель идентификатора с номером от наименования
	\caption{Тестирующий класс \textit{RomanNumeralTest}}
	% окружение учитывает пробелы и табуляции и применяет их в сответсвии с настройками
	\begin{lstlisting}[language={Java}]
		import static org.junit.jupiter.api.Assertions.*;
		import org.junit.jupiter.api.Test;
		
		public class RomanNumeralTest {
			
			@Test
			void convertSingleDigit() {
				RomanNumeral roman = new RomanNumeral();
				int result = roman.convert("C");
				
				assertEquals(100, result);
			}
			
			@Test
			void convertNumberWithDifferentDigits() {
				RomanNumeral roman = new RomanNumeral();
				int result = roman.convert("CCXVI");
				
				assertEquals(216, result);
			}
			
			@Test
			void convertNumberWithSubtractiveNotation() {
				RomanNumeral roman = new RomanNumeral();
				int result = roman.convert("XL");
				
				assertEquals(40, result);
			}
		}
	\end{lstlisting}
\end{ListingEnv}

