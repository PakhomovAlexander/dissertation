\chapter{Анализ задачи} 

\fixme{
	Во второй главе приводится постановка задачи, ее содержательное и формализованное описание.
	Например, если работа связана с разработкой информационных си- стем и использованием информационных технологий, в содержательной постановке приводятся ссылки на документы, регламентирующие процесс функционирования информационной системы, основные показатели, ко- торые должны быть достигнуты в условиях эксплуатации информационной системы; ограничения на время решения поставленной задачи, сроки выдачи информации, способы организации диалога человека с инфор- мационной системой средствами имеющегося инструментария, описание входной и выходной информации (форма представления сообщений, описа- ние структурных единиц, периодичность выдачи информации или частота поступления), требования к организации сбора и передачи входной инфор- мации, ее контроль и корректировка.
	В математической постановке (при наличии) выполняется формали- зация задачи, в результате которой определяется состав переменных, кон- стант, их классификация, виды ограничений на переменные и математиче- ские зависимости между переменными. Устанавливается класс, к которому относится решаемая задача, и приводится сравнительный анализ методов решения для выбора наиболее эффективного метода. Приводится обоснова- ние выбора метода решения.
	Вместо математической модели для формализации задачи может быть выбран любой иной вид моделей, в том числе функциональные, информа- ционные, событийные, структурные. Могут быть представлены модели «как есть» и «как должно быть». В этом случае также следует предложить спо- собы перехода.
	В целом, во второй главе определяется общая последовательность решения задачи. Здесь же приводятся результаты теоретических исследова- ний.
	Описание разработанных алгоритмов, анализ их эффективности мо- жет присутствовать как во второй главе, так и вынесено в отдельную главу (алгоритмическое обеспечение). Все зависит от объема представляемого материала.
}