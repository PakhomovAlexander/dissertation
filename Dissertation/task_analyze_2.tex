\chapter{Анализ  задачи} 

\section{Постановка задачи}

Проанализировать существующие инструменты автогенерации модульных тестов. Разработать инструмент автогенерации модульных тестов для Java программ. Инструмент должен удовлетворять следующим требованиям:

\begin{itemize}
	\item поставляться в виде отдельной библиотеки;
	\item для указанного Java класса генерировать соответствующий модульный тест или несколько модульных тестов;
	\item набор тестовых сценариев должен обеспечивать максимально возможное покрытие по критерию покрытия веток исполнения;
	\item сгенерированный код должен быть понятен человеку.
\end{itemize}


\section{Обзор существующих инструментов автогенерации модульных тестов}

Автогенерация модульных тестов~--- свежая область компьютерных наук. Большинство инструментов автогенерации кода используются исследователями и служат примерами идей, таких как фаззинг или SBST. 

\subsection{Randoop}

Randoop~--- первый инструмент автогенерации модульных тестов для Java программ~[14]. Сценарий использования: 

\begin{enumerate}
	\item скачать архив;
	\item разархивировать скаченный файл;
	\item указать соответствующие переменные окружения RANDOOP\_PATH и RANDOOP\_JAR;
	\item запустить java программу и указать набор классов для генерации.
\end{enumerate}

Преимущества Randoop: 

\begin{itemize}
	\item открытый исходный код;
	\item лицензия, не ограничивающая коммерческое использование;
\end{itemize}

Недостатки:

\begin{itemize}
	\item сложность использования;
	\item низкое качество сгенерированного кода: сложно читать, слабое покрытие;
	\item не поддерживаются версии Java, выше 11;
	\item не поддерживается библиотека для тестирования JUnit версии 5.
\end{itemize}

\subsection{EvoSuite}

EvoSuite~--- инструмент автогенерации модульных тестов, который используется исследователями для проверки 
гипотез и разработки новых алгоритмов автогенерации кода~[15].

Сценарий использования: 

\begin{enumerate}
	\item скачать jar файл;
	\item запустить java программу и указать набор классов для генерации.
\end{enumerate}

Преимущества EvoSuite: 

\begin{itemize}
	\item открытый исходный код;
	\item высокое качество сгенерированного кода;
\end{itemize}

Недостатки:

\begin{itemize}
	\item лицензия, ограничивающая коммерческое использование;
	\item сложность использования;
	\item не поддерживаются версии Java, выше 11;
	\item не поддерживается библиотека для тестирования JUnit версии 5.
\end{itemize}

\clearpage

\subsection{Cover}

Cover~--- коммерческий продукт, разработанный компанией Diffblue~[16]. Является самым удобным и качественным инструментом
на~данный момент.

Сценарий использования: 

\begin{enumerate}
	\item скачать IntelliJ IDEA плагин или консольную программу;
	\item запустить консольную программу или нажать на соответствующую кнопку в idea.
\end{enumerate}

Преимущества Cover: 

\begin{itemize}
	\item высокое качество сгенерированного кода;
	\item поддержка современных библиотек, таких как Spring;
	\item удобство использования.
\end{itemize}

Недостатком инструмента является его платная основа.


Сравнительный анализ описанных ранее инструментов приведен в~табл.~\ref{analys}.
\begin{table} [h!tbp]
	\centering
	\changecaptionwidth\captionwidth{14.75cm}
	\caption{Сравнительный анализ инструментов}\label{analys}%
	\begin{tabular}{| p{7cm} | p{2cm} | p{2cm} | p{2cm} |} \hline
		\textbf{Инструмент}							 	 		  &	\textbf{Randoop}	&	\textbf{EvoSuite}	&	\textbf{Cover}			\\ \hline
		коммерческое использование  					& 	+						   & 		--						& платно						\\ \hline
		удобство использования 					 			 & -- 							& -- 							 & +						      	\\ \hline
		открытый исходный код 					 			  & +							& +							      & --						      	\\ \hline
		качество сгенерированного кода 			    	 & низкое				  & высокое					   & высокое					\\ \hline
		поддержка новых библиотек  	                       & --				            & --					   		   & +						 		\\ \hline
	\end{tabular}
\end{table}	 
























